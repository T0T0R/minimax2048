\documentclass[a4paper]{report}
\usepackage[utf8]{inputenc}
\usepackage[T1]{fontenc}
\usepackage[francais]{babel}
\usepackage{lmodern} %fonte latin modern
\usepackage{soul} % Barrer un mot \st{}
\usepackage{xcolor}
\usepackage{xspace}% gère les espaces de \of et \fg
\usepackage{graphicx}
\usepackage{wrapfig}
\usepackage{amsmath}
\usepackage{amssymb}
\usepackage{amssymb,amsmath,amsfonts,amsxtra,amsthm}
\usepackage{mathrsfs} % Intégrales
\usepackage{stackrel} % \stackrel[k_2]{k_1}{\rightleftarrows}
\usepackage{esint} % various fancy integral symbols
\usepackage{stmaryrd} % Intervalles discrets |[ ]| avec \llbracket \rrbracket 
\usepackage{booktabs} % Pour les tableau : Allows the use of \toprule, \midrule and \bottomrule in tables for horizontal lines
\usepackage{physics}
\usepackage{chemist}
\usepackage{chemmacros} %https://la-bibliotex.fr/2019/02/02/ecrire-la-chimie-avec-latex/
\chemsetup{modules=all}
\chemsetup[phases]{pos=sub}
\chemsetup[acid-base]{p-style=slanted}
\usepackage{chemfig} %Dessiner molécules
\usepackage{modiagram} %Molecular diagrams
\usepackage{float} % Pour le placement absolu des figures avec l'option [H]
\usepackage{listings} % Pour le code
\usepackage{subfigure}
\usepackage{url}
\usepackage[top=2cm, bottom=2cm, left=1cm, right=1cm]{geometry}
\usepackage{tikz,tkz-tab}
\usepackage{setspace}


\newcommand*{\Coord}[3]{%Pour donner les coordonnées d'un vecteur : \Coord{u}{X}{Y} 
  \ensuremath{\overrightarrow{#1}\, 
    \begin{pmatrix} 
      #2\\ 
      #3 
    \end{pmatrix}}}

\newcommand*{\E}{%Donne le symbole de l'énergie E rond
	\mathcal{E} 
}

\newcommand\angstrom{\mbox{\normalfont\AA}}

\newcommand\namebond[4][5pt]{\chemmove{\path(#2)--(#3)node[midway,sloped,yshift=#1]{#4};}}

\newcommand\arcbetweennodes[3]{%
	\pgfmathanglebetweenpoints{\pgfpointanchor{#1}{center}}{\pgfpointanchor{#2}{center}}%
	\let#3\pgfmathresult
}

\newcommand\arclabel[6][stealth-stealth,shorten <=1pt,shorten >=1pt]{% Trace et nomme des angles {rayon,depart,arrivee,centre,text}
	\chemmove{%
		\arcbetweennodes{#4}{#3}\anglestart \arcbetweennodes{#4}{#5}\angleend
		\draw[#1]([shift=(\anglestart:#2)]#4)arc(\anglestart:\angleend:#2);
		\pgfmathparse{(\anglestart+\angleend)/2}\let\anglestart\pgfmathresult
		\node[shift=(\anglestart:#2+1pt)#4,anchor=\anglestart+180,rotate=\anglestart+90,inner sep=0pt,
			outer sep=0pt]at(#4){#6};
	}
}

\title{\bsc{Data Sciences} - Rapport de projet :\\ IA : Algorithme MinMax appliqué au 2048}
\author{\bsc{Langlard} Arthur, \bsc{Marcone} Jules}
\date{\today}



\begin{document}
\lstdefinestyle{customc}{
  belowcaptionskip=1\baselineskip,
  breaklines=true,
  frame=L,
  xleftmargin=\parindent,
  language=C++,
  showstringspaces=false,
  basicstyle=\footnotesize\ttfamily,
  keywordstyle=\bfseries\color{green!40!black},
  commentstyle=\itshape\color{purple!40!black},
  identifierstyle=\color{blue},
  stringstyle=\color{orange},
}
\lstdefinestyle{matlab}{
  belowcaptionskip=1\baselineskip,
  breaklines=true,
  frame=L,
  xleftmargin=\parindent,
  language=matlab,
  showstringspaces=false,
  basicstyle=\footnotesize\ttfamily,
  keywordstyle=\bfseries\color{green!40!black},
  commentstyle=\itshape\color{purple!40!black},
  identifierstyle=\color{blue},
  stringstyle=\color{orange},
}




\maketitle

\chapter{Code fourni}
On nous a fourni les fonctions \verb|fournir_coups|, \verb|glisse|, \verb|afficher_position| et \verb|pressee|.
Le jeu 2048 est open-source, ce qui nous a permis d'adapter son code en C++ et en Matlab.

\section{Fonctions fournies}
\subsection{fournir\_coups}
\verb|fournir_coups| est une fonction ayant pour paramètres : 
\begin{itemize}
\item \verb|position| : Un objet contenant l'état de la partie sous la forme d'une grille de 4$\times$4 cellule,
\item \verb|trait| : \emph{Vrai} si le prochain coup à évaluer est un déplacement des tuiles, \emph{faux} si le prochain coup à évaluer est l'insertion d'une tuile 2 ou 4.
\end{itemize}
Cette fonction retourne \verb|liste_coups|, un tableau de grilles correspondant à toutes les grilles qu'il est possible d'obtenir à partir de la situation décrite par l'objet \verb|position| et suivant la valeur de \verb|trait|.

\paragraph{Matlab}
\lstinputlisting[style=matlab, caption={fournir\_coups.m}]{./resources/fournir_coups.m}
\begin{enumerate}
\item Le tableau \verb|liste_coups| est initialisé,
\item on vérifie que la note attribuée à la grille actuelle \verb|position.M| soit positive,
\item dans le cas d'un déplacement des tuiles, on stocke successivement dans \verb|liste_coups| chacune des quatre grilles possibles après un déplacement vers la gauche, la droite, le haut, le bas, en vérifiant qu'elle ne soit pas déjà présente dans le tableau,
\item dans le cas d'une insertion de tuile, chaque case vide de la grille actuelle est évaluée, en la remplaçant par un 2 ou un 4. Les grilles obtenues sont ajoutées à \verb|liste_coups|.
\item Finalement, le tableau de grilles possibles \verb|liste_coups| est mélangé puis retourné.
\end{enumerate}

\paragraph{C++}
\lstinputlisting[style=customc, caption={Extrait de provided.cpp}]{./resources/fournir_coups.cpp}
\begin{enumerate}
\item La grille correspondant à la position initiale est stockée dans \verb|grilleActuelle|,
\item Le tableau \verb|liste_coups| est initialisé,
\item on vérifie que la note attribuée à la grille actuelle \verb|grilleActuelle| soit positive,
\item les différents mouvements sont stockés dans \verb|directions| et les valeurs possibles d'une tuile insérée sont stockées dans \verb|valeursTuile|,
\item dans le cas d'un déplacement des tuiles, on stocke successivement dans \verb|liste_coups| chacune des quatre grilles possibles après un déplacement en vérifiant qu'elle ne soit pas déjà présente dans le tableau,
\item dans le cas d'une insertion de tuile, chaque case vide de la grille actuelle est évaluée parmi ses valeurs possibles. Les grilles obtenues sont ajoutées à \verb|liste_coups|.
\item Finalement, le tableau de grilles possibles \verb|liste_coups| est retourné.
\end{enumerate}

\paragraph{}On a également défini une fonction \verb|isInside| avec pour paramètres le tableau \verb|liste_coups| et la grille à rechercher dedans \verb|coup|.
\begin{enumerate}
\item La sortie \verb|output| est initialisée à \emph{false} par défaut.
\item On regarde chaque élément de \verb|lise_coups|,
\item si cet élément est identique à \verb|coup|, la variable \verb|output| est passée à \emph{true} et on sort de la boucle.
\item La fonction retourne \verb|output|. 
\end{enumerate}




\subsection{glisse}
\verb|glisse| est une fonction ayant pour paramètres : 
\begin{itemize}
\item \verb|position| : Un objet contenant l'état de la partie sous la forme d'une grille de 4$\times$4 cellules,
\item \verb|fleche| : Une chaîne de caractère indiquant la direction de glissement des tuiles (g,d,h,b)
\end{itemize}
Cette fonction retourne \verb|position|, l'objet contenant l'état de la partie sous la forme d'une grille une fois le glissement effectué dans la direction de \verb|fleche|. En C++, on retournera simplement la matrice représentant la grille plutôt que l'objet contenant cette grille.

\paragraph{Matlab}
\lstinputlisting[style=matlab, caption={glisse.m}]{./resources/glisse.m}
\begin{enumerate}
\item La matrice \verb|M| représentant le plateau de la partie est récupérée,
\item suivant la valeur de \verb|fleche|, la matrice \verb|M| est temporairement pivotée pour pouvoir appliquer le même algorithme de glissement (vers le bas) des tuiles, quelque soit la direction de ce glissement.
\item La dernière ligne ne peut pas être plus glissée vers le bas, on ne s'intéresse donc qu'aux lignes \verb|i|$\in\llbracket1;3\rrbracket$. Plus précisément, la priorité de la combinaison de deux tuiles se fait avec d'abord la tuile la plus basse, puis celle au dessus, etc. Donc en réalité, \verb|i| varie dans le sens \verb|i|$\in\llbracket3;1\rrbracket$ (en remontant).
\item La possibilité d'une combinaison est évaluée en observant si à la ligne \verb|i| il y a une case non-vide, et si en-dessous (à la ligne \verb|ii|) il y a une case contenant la même valeur. Si c'est le cas, les deux tuiles sont combinées, sinon la tuile en \verb|i| est simplement déplacée au-dessus de \verb|ii|.
\item On répète l'algorithme pour les \verb|j|$\in\llbracket1;4\rrbracket$ colonnes de \verb|M| pivotée.
\item La matrice \verb|M| est remise dans sa position initiale, puis \verb|position| est mis à jour avec \verb|M| et est retourné.
\end{enumerate}

\paragraph{C++}
\lstinputlisting[style=customc, caption={Extrait de provided.cpp}]{./resources/glisse.cpp}
\begin{enumerate}
\item La matrice \verb|M| représentant le plateau de la partie est récupérée,
\item suivant la valeur de \verb|fleche|, la matrice \verb|M| est temporairement pivotée pour pouvoir appliquer le même algorithme de glissement (vers le bas) des tuiles, quelque soit la direction de ce glissement.
\item La dernière ligne ne peut pas être plus glissée vers le bas, on ne s'intéresse donc qu'aux lignes \verb|i|$\in\llbracket0;2\rrbracket$. Plus précisément, la priorité de la combinaison de deux tuiles se fait avec d'abord la tuile la plus basse, puis celle au dessus, etc. Donc en réalité, \verb|i| varie dans le sens \verb|i|$\in\llbracket2;0\rrbracket$ (en remontant).
\item La possibilité d'une combinaison est évaluée en observant si à la ligne \verb|i| il y a une case non-vide, et si en-dessous (à la ligne \verb|scanner|) il y a une case contenant la même valeur. Si c'est le cas, les deux tuiles sont combinées, sinon la tuile en \verb|i| est simplement déplacée au-dessus de \verb|scanner|.
\item On répète l'algorithme pour les \verb|j|$\in\llbracket0;3\rrbracket$ colonnes de \verb|M| pivotée.
\item La matrice \verb|M| est remise dans sa position initiale puis est retournée.
\end{enumerate}



\subsection{afficher\_position}
\verb|afficher_position| est une fonction ayant pour paramètre \verb|position|, un objet contenant l'état de la partie sous la forme d'une grille de 4$\times$4 cellules.

Cette fonction permet d'afficher à l'utilisateur la partie.

\paragraph{Matlab}
\lstinputlisting[style=matlab, caption={afficher\_position.m}]{./resources/afficher_position.m}
\begin{enumerate}
\item La matrice \verb|M| représentant l'état de la partie est récupérée de \verb|position|.
\item Une table associative \verb|Coul| permettant d'attribuer à chaque valeur de tuile rencontrée une couleur au format RGB est définie.
\item Un rectangle délimitant l'espace de jeu et tracé,
\item puis pour chaque case de \verb|M|, sa valeur \verb|n| est écrite à sa position dans un petit rectangle de couleur définie dans \verb|Coul|.
\item Enfin, la figure est affichée
\end{enumerate}


\subsection{pressee}
\verb|pressee| est une fonction permettant au programme d'écouter l'entrée clavier. Elle retourne une chaîne de caractère correspondant à la touche pressée par l'utilisateur.

\paragraph{Matlab}
\lstinputlisting[style=matlab, caption={pressee.m}]{./resources/pressee.m}
\begin{enumerate}
\item Une chaîne de caractères vide \verb|fleche| est initialisée,
\item puis le clavier de l'utilisateur est écouté, et la valeur ASCII de la touche pressée est stockée dans \verb|value|.
\item Si la touche pressée correspond à une des touches de direction ou la touche s, le caractère correspondant est stocké dans \verb|fleche| (respectivement g,d,h,b,s).
\item La boucle recommence si \verb|fleche| est vide, sinon \verb|fleche| est renvoyée par la fonction.
\end{enumerate}


\section{Sources de 2048}
Le dépôt officiel des sources du jeu 2048 est \url{https://github.com/gabrielecirulli/2048} sous licence MIT.

Les fichiers constituant le noyau du jeu sont \verb|game_manager.js|, \verb|grid.js| et \verb|tile.js|.

\subsection{GameManager}
\paragraph{C++}
\lstinputlisting[style=customc, caption={GameManager.cpp}]{./src/GameManager.cpp}
\lstinputlisting[style=customc, caption={GameManager.h}]{./src/GameManager.h}


\subsection{Grid}
\paragraph{C++}
\lstinputlisting[style=customc, caption={Grid.cpp}]{./src/Grid.cpp}
\lstinputlisting[style=customc, caption={Grid.h}]{./src/Grid.h}


\subsection{Tile}
\paragraph{C++}
\lstinputlisting[style=customc, caption={Tile.cpp}]{./src/Tile.cpp}
\lstinputlisting[style=customc, caption={Tile.h}]{./src/Tile.h}



\chapter{Code ajouté}
\section{Fichier principal - mm2048}
\subsection{C++}
On définit une structure \verb|Position| qui représente un couple d'entiers $(x,y)$ modifiables.

\lstinputlisting[style=customc, caption={main.cpp}]{./src/main.cpp}
\lstinputlisting[style=customc, caption={mm2048.h}]{./src/mm2048.h}

\section{Vec2D}
La classe \verb|Vec2D| a été ajoutée en C++ pour permettre de représenter facilement une matrice 4$\times$4 de manière unique. C'est cet objet qui est utilisé dans les fonctions \verb|fournir_coups| et \verb|glisse|. Il inclut une conversion explicite depuis un objet \verb|Grid| ainsi que depuis des tableaux 4$\times$4 de \verb|int| ou de \verb|Tile|.

Il s'agit formellement d'un objet \verb|std::array<std::array<int,4>,4>|.

Plusieurs opérateurs sont surchargés:
\begin{itemize}
\item == : Permet la comparaison de deux matrices (A==B)
\item + : Permet l'addition de deux matrices (A+B)
\item * : Permet la multiplication d'une matrice par un coefficient (k*A ou A*k)
\item / : Permet le produit matriciel de deux matrices (A/B)
\end{itemize}

Il existe la méthode \verb|eye| pour générer une matrice identité 4$\times$4.

\lstinputlisting[style=customc, caption={Vec2D.cpp}]{./src/Vec2D.cpp}
\lstinputlisting[style=customc, caption={Vec2D.h}]{./src/Vec2D.h}














\tableofcontents
\lstlistoflistings
\end{document}